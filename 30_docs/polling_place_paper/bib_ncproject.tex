\documentclass[12pt]{article} 
\usepackage[pdftex]{graphics}
\usepackage[english]{babel}
\usepackage{graphicx}
\usepackage{float}
\usepackage{chngpage}
\usepackage{multirow, lscape}
\usepackage{multicol}
\usepackage{url}
\usepackage{color}
\usepackage[bookmarks=true, bookmarksopen=true, linkcolor=webblue]{hyperref}
\usepackage[left=1in,top=1in,right=1in,bottom=1in,nohead]{geometry}
\usepackage{booktabs, multicol, multirow}
\usepackage[doublespacing]{setspace}
\usepackage{tocloft}
\renewcommand{\baselinestretch}{2.0} %This sets the line spacing, right now it is set at 1.5 spacing. 
\usepackage{natbib}
\usepackage{eqnarray,amsmath}

\title{Bibliography: North Carolina Project}

\begin{document}
\maketitle

\begin{enumerate}

\item \citet{herron2015race} Law Review article. Analyzes the racial impacts of the Voter Information Verification Act, passed in NC following the \textit{Shelby County} decision. They show that, ``black early voters in North Carolina have in presidential elections cast their ballots disproportionately in the first week of early voting, an early voting week that VIVA has eliminated; that blacks in the state disproportionately have registered to vote during early voting and in the immediate run-up to Election Day, something that VIVA prohibits; that North Carolina registered voters who lack two VIVA-acceptable forms of voter identification, drivers licenses and nonoperator identification cards, are disproportionately black; that VIVA?s identification dispensation for voters at least 70 years is a disproportionate benefit to whites; and, that preregistered 16 and 17-year old voters in North Carolina, a category of registrants that VIVA prohibits, are disproportionately black." 

\item \citet{herron2016precinct} article that uses data from 2014 Federal election in Hanover, NH to simulate how precinct resources affect wait times. The authors propose a new method for studying wait times. 

\item \citet{herron2015precinct} studies the effect of varied closing times in the 2012 Florida general election. Precinct closing times are meant to be proxies for polling place congestion/wait times. Use data from over 5,000 precincts in Florida. Precincts with more Hispanic voters closed later. Suggest that closing times/congestion should be used to study racial variation in barriers to voting. 

\item \citet{herron2014race} study of change in Florida law regarding early voting time windows. Florida reduced the length of the early voting window for the 2012 election. They compare individuals' voting patterns from 2008 to 2012 and find that, `` find that racial/ethnic minorities, registered Democrats, and those without party affiliation had significant early voting participation drops and that voters who cast ballots on the final Sunday in 2008 were disproportionately unlikely to cast a valid ballot in 2012." 

\item \citet{stewart2015waiting} reviews evidence of voting lines, suggests that lines discourage voting, lower voter confidence, and impose economic costs on voters. Not much data/anything new here. 

\item \citet{ansolabehere2005residual} analysis of voting technology. Find that counties that use paper ballot have the lowest rates of uncounted votes, then optically scanned ballots, then mechanical level machines, direct register electronic machines, and final punch cards. Results suggest that if all the places that used punch cards used optically scanned ballots instead, 500,000 more votes would have been counted. 

\item \citet{grimmer2017comment} research comment on the infamous 2017 Hajnal et al. study. Suggests that the results of the Hajnal study are a function of data inaccuracies. Finds no evidence of voter identification affecting turnout. 

\item \citet{meredith2014voting} study of ex-felon voting rights notification laws using data from New Mexico, New York, and North Carolina. Find little evidence that ex-felon registration or turnout increases after laws that inform felons of their voting rights are implemented. 

\item \citet{meredith2011convenience} study of the effects of voting-by-mail on election outcomes. Exploits quasi-random variation in state law that sets a threshold for which areas are assigned to VBM and which that aren't. VBM effects outcomes because voters that VBM have a higher probably of voting for candidates that drop out before election day. 

\item \citet{berger2008contextual} articles that examines the impact of what the actual polling location is (church, school, etc. on vote choice). ``People who were assigned to vote in schools were more likely to support a school funding initiative. This effect persisted even when controlling for voters' political views, demographics, and unobservable characteristics of individuals living near schools. A follow-up experiment using random assignment suggests that priming underlies these effects, and that they can occur outside of conscious awareness." 

\item \citet{brady2011turning} one of five existing papers on the impact of changing polling places on elections. Exploits the consolidation of polling places during the 2003 Gubernatorial recall election. Shows an overall decrease in turnout of 1.85\%. The changes decreased polling place turnout by 3.03\%, but increased absentee voting by 1.18\%. No evidence that closings correlated with partisanship, but changes affected Democrats the most. 

\item \citet{verba1995voice} probably drop this in for who we would expect the changes to affect most. 

\item \citet{ansolabehere2012movers} creates a model to try to explain why age is correlated with voting. Show that it ties to mobility habits of young people and registration rules. 

\item \citet{fraga2010voting} uses election day weather as an exogenous cost imposed on voters. Rain decreases turnout on average, but not in competitive elections. Suggests that mobilization efforts or campaign activity in competitive environments helps over come costs. 

\item \citet{fraga2016examining} the Voting Rights Act's language provisions increased the participation and registration rates of Latino's and Asian American's. 

\item \citet{fraga2016redistricting} uses redistricting as a source of variation in the racial context for voters. Finds that Blacks, Whites, and Asian American voters turnout to vote at higher rates when assigned to a co-ethnic member of Congress. 

\item \citet{fraga2016candidates} seems like it contradicts the study above. Finds that minority turnout is not higher in districts with co-ethnic candidates. Instead, minority turnout is higher in areas in which their share of the population is greater. 

\item \citet{gimpel2003political} one of five papers on polling place distance, uses data from three Maryland counties to study impact of distance to polling places. Finds that distance matters, but in a non-linear fashion. Distance matters most for suburban voters, rural voters who face longer routes are not affected. 

\item \citet{dyck2005distance} one of five uses data on Clark County Nevada. Like the study above, finds a non-linear relationship. Distance is also associated with voting by mail. 

\item \citet{hajnal2017voter} finds that voter id laws impact racial minorities more than Whites an that this shifts election results to the right. Study challenged by Grimmer article above. 

\item \citet{highton2017voter} argues that it is hard to estimate the effect of voter id laws. Finds limited evidence and suggests we may need to wait to understand the effect of strict voter ID laws. 

\item \citet{highton2006long} case study of the long lines in Columbus, OH in 2004. Finds that voting machine scarcity contributed a small amount to the long lines. 

\item \citet{highton2004voter} refers to voter registration as the most significant cost that voters face. Provides a history of voter registration laws. 

\item \citet{highton2000residential} attempts to assess why mobility affects turnout. Tests between whether moving impacts turnout because of the need to register again or because of lost social connectedness. Finds evidence of both. However, registration appears to be the most significant cost. 

\item \citet{highton1997easy} voter registration laws are not solely responsible for the education gap in turnout. 

\item \citet{burden2014election} election day registration is consistently associated with higher turnout; early voting is associated with lower turnout. 

\item \citet{burden2017complicated} early voting helps Republicans, not Democrats. 

\item \citet{burden2013selection} tests the difference in behavior from elected and appointed local election officials. They find that elected officials are more concerned with access rather than security. Areas with appointed officials have lower turnout when the electorate differs in partisanship from the official. 

\item \citet{burden2013election} voter registration reduces turnout, especially in areas with lower election administration capacity. 

\item \citet{rosenstone1993mobilization} classic on role of elites in mobilization. 

\item \citet{wolfinger1980votes} socio-economic status and registration

\item \citet{rosenstone1978effect} registration laws lower turnout. 

\item \citet{anzia2011election} off cycle elections have lower turnout, allows organized groups to gain advantage. 

\item \citet{gerber2008social} field experiment for turnout. Social pressure increases turnout. 

\item \citet{hanmer2009discount} assess the effect of voter registration. Finds that policies meant to increase turnout help modestly. Stringent policies reduce turnout modestly. 

\item \citet{leighley2013votes} replication and extension of \textit{Who Votes?}. Finds similar answers. Voter registration policies have small, but meaningful effects. Finds that non-voters are more liberal on economic policy than voters. 

\item \citet{mcdonald2001myth} gives us VEP and argues that turnout is not declining. 

\item \citet{powell1986american} American voter turnout lags behind the rest of the world because of its restrictive registration laws. 

\item \citet{riker1968theory} costs and benefits (a la civic duty) of voting. 

\item \citet{white2015need} cool field experiment on local election administration. Uses Latino sounding names to assess whether election administrators discriminate against racial and ethnic minorities. Find that Latinos are less likely to receive a response and that response is of lower quality than responses to non-Hispanic Whites. 

\item \citet{haspel2005location} one of the five studies on polling places. Studies Atlanta. Increased distances reduces turnout. 

\item \citet{barreto2009disproportionate} racial minorities, the poor, and the young are more likely to be affected by photo ID requirements. These people are disproportionately Democrats. 

\item \citet{barreto2009all} study of Los Angeles. Low-income and minority areas have polling places that are harder to find and of lower quality. This leads to lower voter turnout. 

\item \citet{hicks2015principle} looks at the passage of voter ID legislation. Republicans are more likely to pass voter ID laws, but mostly in competitive places or places where they are losing control over state. 

\item \citet{citrin2014effects} counter-mobilization efforts against voter ID laws increases turnout (or reduces the negative effects of Voter ID). 

\item \citet{downs1957} costs and benefits. 

\item \citet{berman2016} POPULAR PRESS. Discussion of reductions in polling places following \textit{Shelby}. 

\item \citet{roth2015} POPULAR PRESS. talks about impact of polling places changes on African Americans in North Carolina, post 2014 election. 

\item \citet{ap2018} POPULAR PRESS. discussion of battle between North Carolina state legislature and governor over control of the State election board. 


\item \citet{vasilogambros2018} POPULAR PRESS. Discussion of racial disparities in voting lines. 

\item \citet{gargan2017} POPULAR PRESS. Talks about how Wake County increased turnout via opening more polling places. 

\item \citet{oliver1996effects} early voting only increased turnout in places where party's mobilized supporters. 

\item \citet{nagler1991effect} registration laws do not differentially impact people of different educational backgrounds. 

\item \citet{hershey2009we} review article that discusses history and differential impacts of voter registration laws (no new findings). 

\item \citet{alvarez2008effect} Not much evidence of an effect of voter ID laws. The strictest forms lower turnout more than less strict ones. Effects mostly isolated with minorities and low socio-economic status individuals. 

\item \citet{atkeson2010new} focuses on New Mexico. Hispanic, male, and election day voters are more likely to be required to present IDs than others. No evidence of poll worker discrimination though. 

\item \citet{cantoni2016} one of five papers on polling place distance;  comparison of census blocks near borders in Massachusetts and Minnesota. Distance to polling negatively related to turnout. Effects are especially large in predominantly minority areas. 



\item \citet{amos2017reprecincting} newest paper on impact of polling place changes. Focuses on manatee county fl. Selects on the DV. Finds decline in turnout. First to consider partisan targeting, but doesn't appropriately model the relationship. Again relies on single county, preventing us from understanding the over all election impact. 

\item \citet{michelson2012effect} field experiment with election administrators. Some voters got postage stamps for vote by mail and others didn't. Providing stamps did not increase voting by mail, instead it increased voting in person, but produced no change in aggregate turnout. Effects are largest for people that formerly voted by mail because it disrupted their routine and confused voters. 


\item \citet{gerber2013identifying} using variation in county-level implementation of all-mail elections in Washington State, they find that vote by mail increased turnout by 2-4 points. The effects were largest for lower participators. 

\item \citet{arceneaux2012get} campaign mobilization really only impacts people that vote in person, not people that vote by mail. 

\item \citet{kousser2007does} vote by mail does not increase turnout. However, vote by mail increases turnout in low participation contests (BY SIMILAR LOGIC MAYBE THIS IS WHY WE SEE LARGE EFFECTS OF POLLING PLACE CHANGES WHEN LOOKING AT SINGLE COUNTIES IN WEIRD ELECTIONS). 

\item \citet{plutzer2002becoming} important article on habitual voting. When people become eligible to vote, their parental resources shape early participation patters. Inertia then takes over and people get on different trajectories. Parental influence declines over the life cycle. As people age, more become habitual voters, as disruptions to habits and life become fewer. 


\item \citet{malhotra2011text} habitual voters are most likely to be impacted by mobilization in lower salience elections and casual voters are most likely to be impacted in higher salience elections. 

\item \citet{squire1987residential} moving reduces turnout. Finds that turnout would increase by 9\% if the impact of moving could be remove. Almost exactly the amount we observe in the disparities in our sample. 

\item \citet{cho2006residential} neighborhood socialization can increase or decrease participation. Social context is important. 


\item \citet{brody1977life} voting is a habit. 

\item \citet{gerber2003voting} voting is habit forming. (WE CAN USE PANEL STRUCTURE TO ADD SOME NUANCE TO UNDERSTANDINGS OF HABIT FORMATION IN VOTING). 

\item \citet{pettigrew2017racial} african Americans face longer lines within counties than white voters. 

\item \citet{karp2000going} vote by mail modestly increased turnout, but mostly only in low salience elections. The increase came mostly among people pre-disposed to vote, such as high socioeconomic status. 


\item \citet{gronke2008psychological} early voters are more likely to be highly educated and more cognitively engaged in the campaign. 

\item \citet{gronke2007early} only vote by mail seems to increase turnout among all of the other reforms.

\item \citet{arceneaux2009mobilized} GOTV drives mostly increase the turnout of people who are right at the indifference line between voting and not-voting. They do not exist to mobilize non-voters. Only in high turnout elections can these people be mobilized. 

\item \citet{doherty2017voting} social norms, not-battle ground status or pivotality, shape turnout. 

\item \citet{gerber2017generalizability} most social pressure experiements are performed in low-salience elections. re-tests for generalizability. Large scale field experiement across 17 states finds that the effects are generalizable. 



Is there anything that suggests younger voters are more receptive to GOTV efforts and mailers?  That might help explain the age results?




\end{enumerate}




\newpage

\bibliographystyle{apsr}
\bibliography{precinct.bib}

\end{document}

