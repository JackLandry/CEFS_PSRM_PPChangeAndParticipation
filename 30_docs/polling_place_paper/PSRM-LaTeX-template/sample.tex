\documentclass{cup_PSRM}
% Use the options 'crop' to show crop marks, and
% 'doublespaced' for double line spacing.
% e.g. \documentclass[crop,doublespaced]{cup_PSRM}
%
% Please note that this template provides a set of environment
% definitions for  "example", "proof", " definition" & "remark"
% and hence is incompatible with packages such as amsthm.
%
\usepackage[utf8]{inputenc}
\usepackage{graphicx}

\addbibresource{sample.bib}

\begin{document}

\markboth{BÄCK AND LINDVALL}{Fiscal Policies Of Multiparty
Governments}

\journalname{Draft Submission to Political Science Research and Methods}

\journalcopy{The European Political Science Association, 2016}
\fpage{X}
\lpage{XXX}
\journalvolume{6}
\journalissue{3}
\doinumber{doi:10.1017/psrm.2016.XX}

\title{Template for Submitting Your Article to PSRM Using Overleaf
%
\thanks{Hanna Bäck is
Associate Professor of Political Science, Lund University, P.O. Box
52, 221 00 Lund, Sweden (hanna.back@svet.lu.se). Johannes Lindvall
is Professor of Political Science, Lund University, P.O. Box 52, 221
00 Lund, Sweden (johannes.lindvall@svet.lu.se). We are very grateful
to Despina Alexiadou, Michael Becher, Marius Busemeyer, Robert
Franzese, Peter Santesson, and the PSRM reviewers and editors for
very helpful comments on previous versions of this article. We also
wish to thank Holger Döring and Albert Falcó-Gimeno for their
advice and support when generating the data used in the article.
Hanna Bäck acknowledges support from the Research Center (SFB) 884
`Political Economy of Reforms' (Project C3), funded by the German
Research Foundation (DFG). Johannes Lindvall acknowledges support
from the European Research Council (Starting Grant No. 284313). To
view supplementary material for this article, please visit
http://dx.doi.org/10.1017/psrm.2014.11}}

\author{HANNA BÄCK \AND JOHANNES LINDVALL}

\maketitle

\begin{abstract}
\dropping{M}any political scientists and economists have argued that coalition governments tend
to accumulate more debt than single-party governments do, but the evidence for this
proposition is mixed. This article argues that only some coalition governments are
more likely to increase public debt than single-party governments: those in which parties are
unable to make credible promises to their partners about future policy. It introduces the concept
of `commitment potential' within coalitions and proposes a way of measuring it. The study
evaluates its theoretical claims using data on 20 advanced democracies observed over a period
of almost 50 years. It finds that multiparty governments with high commitment potential do not,
on average, accumulate more debt than single-party governments, but that governments with
low commitment potential do.
\end{abstract}

\dropping{T}he scholarly debate that began with Roubini's and Sachs's claim that ``the size and
persistence of budget deficits in the industrial countries y is greatest where there have
been divided governments'' (1989, 905--8) has generated a vast literature in economics and
political science. Yet there is little agreement on the nature of the relationship between
multiparty government and debt. In fact, although the received wisdom in much of the
literature is that there is a positive association between coalition government and
increasing debt, even this basic
empirical finding has been disputed in a number of studies, including early contributions
by Edin and Ohlsson (1991) and de Haan and Sturm (1997). In recent years, moreover,
debt has increased the most in countries with a history of single-party governments, such
as Greece and the United Kingdom (Nyman 2012).
The proposition that coalition governments tend\footnote{footnote text}

\begin{hypolist}
\item Coalition government is associated with larger year-to-year increases (or
lower year-to-year decreases) in debt when the government's
commitment potential is low. When commitment potential is high,
coalition governments pursue fiscal policies that are similar to
those of single-party governments.
\end{hypolist}

\section{conflicts and compromises in coalition governments}

When does a party have reason to fear that its current coalition partners might ``betray'' it
by coalescing with other parties (or by forming governments on their own) in the future?
And when, in contrast, do governments have high ``commitment potential,'' since the risk
of betrayal is low? These questions clearly need answers before we examine the
relationship between commitment potential and changes in debt.

The rich literature on coalition formation has identified a number of features of parties
and party systems that are likely to affect the outcomes of government-formation
processes. For example, parties that
are large, centrally located, and have been in government before are more likely to enter
government than small, peripheral and inexperienced parties. At the party-system level, potential governments that are of minimalwinning
size, consist of ideologically similar parties and have formed before (especially
incumbent governments) are more likely to emerge than governments that do not have
these characteristics.
\begin{equation}
\label{secondstage}
O_t = \beta_0 + \beta_1 A_{m-1} + \beta_2 N_{t} + \textbf{X}_t \beta^{\sim}  + \varepsilon_t
\end{equation}
But it is not straightforward to apply these important insights when we attempt to
conceptualize and measure the sort of mutual dependence among parties that we are
concerned with here. We have therefore chosen to create a measure of commitment
potential that is based on historical patterns of cooperation among parties. This is, in our
view, a simple yet valid solution to the conceptualization-and-measurement problem that
we face. First of all, parties very likely look to the past when they attempt to predict the
future behavior of their current partners. Second, historical patterns of coalition formation
should reflect the underlying regularities identified in the coalition-formation literature.
\begin{table}
\centering
\caption{Descriptive Statistics}
\begin{tabular}{@{}lllll@{}}\\\hline
Variable&Min.&Max.&Mean&SD\\\hline
Debt&0.033&1.676&0.434&0.293\\
GDP~Growth&20.089&0.123&0.025&0.027\\\hline
\end{tabular}
\tabnote{Descriptive statistics for all variables except for the fiscal rules variables are based on the sample
used in Model 4, Table 2 (N5883). The descriptive statistics for the fiscal rules variables are based on
the sample used in Model 6 (N5457).}
\end{table}

\begin{enumerate}
\item By so doing Vico
emphasized that the history of civil institutions could not be
separated from the effortful human activity of describing, arguing
and legitimating political interests and activities.

\item By so doing Vico
emphasized that the history of civil institutions could not be
separated from the effortful human activity of describing, arguing
and legitimating political interests and activities.
\end{enumerate}

But it is not straightforward to apply these important insights when we attempt to
conceptualize and measure the sort of mutual dependence among parties that we are
concerned with here. We have therefore chosen to create a measure of commitment
potential that is based on historical patterns of cooperation among parties. This is, in our
view, a simple yet valid solution to the conceptualization-and-measurement problem that
we face. First of all, parties very likely look to the past when they attempt to predict the
future behavior of their current partners. Second, historical patterns of coalition formation
should reflect the underlying regularities identified in the coalition-formation literature.
\begin{figure}%[1]
% \centerline{\fbox{\hbox to 20pc{\vbox to
% 6pc{\hsize20pc{\vfill\hfil\textbf{FPO}\hfil\vfill}}}}}
\centering
\includegraphics[width=6cm]{example-image}
\caption{Debt and coalitions, 1960-2008}
\fignote{the solid lines represent government debt over GDP. The gray areas show periods of coalition
government. The dashed lines describe the commitment potential of all governments.}
\end{figure}
Our dependent variable, year-to-year changes in public debt, is based on a measure of
gross central government debt (as a percentage of GDP) from Reinhart and Rogoff (2009).
We chose a measure of central government debt rather than general government debt
since our argument is concerned with the behavior of national-level decision makers.
Since the Reinhart and Rogoff dataset does not include data on all the countries we are
interested in, we have used data from two other sources---Armingeon et al. (2011) and
the IMF Historical Debt Database \citep{abbas2010historical}---to impute missing values. The
Reinhart and Rogoff and Armingeon et al. series are highly correlated (r50.95), as are
the Reinhart and Rogoff series and the IMF series (r50.92). Mixing data from different
sources, as we do here, is potentially risky, but since the three data series are highly
correlated, we believe that the benefits of including as many observations as possible
outweigh the risks. We use a measure of year-to-year changes in debt rather than a
measure of deficits for three reasons: (1) deficits are more easily misrepresented through
creative accounting, (2) it is the preferred measure in most of the literature that we draw
on, and (3) it is available for a long time period.3

Our analysis covers the period from 1961 to 2008 (or from democratization to 2008)
and includes the following countries: Australia, Austria, Belgium, Canada, Denmark
(from 1967), Finland, France, Germany, Greece (from 1976, with a one-year gap in the
1990s because of missing data), Ireland (from 1976), Italy, Japan, the Netherlands, New
Zealand, Norway, Portugal (from 1977), Spain (from 1978), Sweden, Switzerland and the
United Kingdom. This gives us a dataset of approximately 880 country-years.

The data source that we have used to define the two main explanatory variables'
coalition government and commitment potential is the ParlGov database. For both variables, where more than one cabinet was in government
during a specific year, we have chosen to concentrate on the cabinet with the longest
duration during that year. For the coalition variable, we have chosen to use a simple
dummy (with single-party governments as the reference category) rather than a measure
of the number of parties in government or the ideological range within
the government, since these more nuanced measures generate virtually
identical results (although the ideological range within the government is included as a
control variable in some of our models).

Additional example references \citep{RePEc:aea:aecrev:v:81:y:1991:i:5:p:1170-88} and \citep{19995} are included in the sample bibtex bibliography included with this template.

\printbibliography

\end{document}
